% !TeX encoding = UTF-8
% !TeX program = pdflatex

\documentclass[11pt]{article}

\title{{\bf Public Key Management} \\ \bigskip \large HW5 - CNS Sapienza}
\date{2018-11-30}
\author{Andrea Fioraldi 1692419}

\begin{document}
\maketitle

\section{Introduction}

There are many different encoding and structures for public and private keys.
We present an overview about the formats using the most widespreads Public-Key Cryptography Standards \cite{pkcs} versions.

\section{DER Encoding}

DER (Distinguished Encoding Rules) is the most popular of the ASN.1 \cite{asn1} encodings.

The encoding of an object follows this structure:
\begin{enumerate}
    \item Identifier octets
    \item Length octets
    \item Contents octets
    \item End-of-contents octets 
\end{enumerate}

Complex data structures can be binary-encoded following this structure (you can for example convert any data stired in JSON).

DER is widely used in cryptography to encode the data structures exposed in the following section.

A DER file has usually the extension \verb|.der| and contains only the binary data encoded using such format.

\section{PEM Files}

PEM (Privacy-Ehanced Mail) is a standard file format for storing keys and it was introduced in \cite{rfc7468}.
It is used for both public and private keys.
PEM encode this binary information using base64 and so it is an ASCII format.

The structure is quite simple:
\begin{enumerate}
    \item Header: \verb|-----BEGIN| type of cryptographic data \verb|-----|;
    \item Encoded base64 data, generally encoded with DER;
    \item Footer: \verb|-----END| type of cryptographic data \verb|-----|;
\end{enumerate}

Usually, the extension of a PEM file is \verb|.pem|.

Here an example file of an RSA public key:

\begin{verbatim}
-----BEGIN PUBLIC KEY-----
MIIBIjANBgkqhkiG9w0BAQEFAAOCAQ8AMIIBCgKCAQEA3gmCqlQvCaE9eR31msxo
mM0lJarQfO/kkxAFVmXuLlOXsCt3Nroj3qs58CadPRh4kBg+4KgegkXzaQ8EVIAE
eI4WRR3Ku3dVdX8+i7bNuFGlgoSgpIOgAk4s7SNxRWuoUMTIAg1sxpvYzYeTyDdT
2PjWjkZ3H7M2V3TPoVw9GLoIur0l6Z96vp1LXWX4acocvCZRKLltPQAZiB5c9hXc
kKhNcRWde/5gopv7qyYxxzPQqU4spKID6afNDMsJ9ldK18YQcQvnjo4mIIoQdvFT
i7BJLtxhURQjp5CcZrwFT2Fj3V9MNfYS5yRi/fx17ZMlCAFZLbFVwEK7vLdywyZA
3QIDAQAB
-----END PUBLIC KEY-----
\end{verbatim}

\section{Data structures}

Public and Private keys are stored using different data structures.
We present the most used structures from the Public-Key Cryptography Standards.

\subsection{PKCS\#1}

It is the RSA Cryptography Standard defined in \cite{rfc8017}.
It defines the ASN.1 encoding of RSA public and private keys.

The DER structure of a public key is the following:

\begin{verbatim}
RSAPublicKey ::= SEQUENCE {
    modulus           INTEGER,  -- n
    publicExponent    INTEGER   -- e
}
\end{verbatim}

The structure of a private key is:

\begin{verbatim}
RSAPrivateKey ::= SEQUENCE {
  version           Version,
  modulus           INTEGER,  -- n
  publicExponent    INTEGER,  -- e
  privateExponent   INTEGER,  -- d
  prime1            INTEGER,  -- p
  prime2            INTEGER,  -- q
  exponent1         INTEGER,  -- d mod (p-1)
  exponent2         INTEGER,  -- d mod (q-1)
  coefficient       INTEGER,  -- (inverse of q) mod p
  otherPrimeInfos   OtherPrimeInfos OPTIONAL
}
\end{verbatim}

\subsection{PKCS\#7}

It is the Cryptographic Message Syntax Standard defined in \cite{rfc2315}.
It is usually employes in Public Keys infrastructures.

An associated file extesion is \verb|.p7b| when using a PEM file containing data structured following the PKCS\#7 specification.

The DER structure of both public and provate key is based on the following structure:

\begin{verbatim}
RecipientInfo ::= SEQUENCE 
{
    version                 INTEGER,
    issuerAndSerialNumber   IssuerAndSerialNumber,
    keyEncryptionAlgorithm  KeyEncryptionAlgId,
    encryptedKey            EncryptedKey
} 
\end{verbatim}

The set of structures that can be chained is large and it is explained in a short summary by Microsoft \cite{mspkcs7}.

\subsection{PKCS\#12}

It is one of the complex formats for storing cryptographical objects.

An associated file extesion is \verb|.pfx| or \verb|.p12| and they are archives containing data structured following the PKCS\#12 specification.

One of the major novelties of this format is that the content can be encrypted in a surgical way in containers calles "SafeBags".

The definition of the structures \cite{pkcs12_struct} is very complex and it was criticized for this in the past.

\subsection{A note on encrypted Private Keys}

It is a commom practice to encrypt the private keys using a simmetric algorithm.
The most used are AES and 3-DES.
The previous exposed formats stores fields in order to recognize if a key is crypted and wich algorithm was used.


\vfill

\begin{thebibliography}{99}


\bibitem{pkcs}
{\em PKCS - Wikipedia}.

  \verb|https://en.m.wikipedia.org/wiki/PKCS|
  \newblock Accessed: 2018-12-6.

\bibitem{asn1}
{\em Abstract Syntax Notation One - Wikipedia}.

  \verb|https://en.wikipedia.org/wiki/Abstract_Syntax_Notation_One|
  \newblock Accessed: 2018-12-6.

\bibitem{rfc7468}
{\em RFC 7468}.

  \verb|https://tools.ietf.org/html/rfc7468|
  \newblock Accessed: 2018-12-6.

\bibitem{rfc8017}
{\em RFC 8017}.

  \verb|https://tools.ietf.org/html/rfc8017|
  \newblock Accessed: 2018-12-6.

\bibitem{rfc2315}
{\em RFC 2315}.

  \verb|https://tools.ietf.org/html/rfc2315|
  \newblock Accessed: 2018-12-6.

\bibitem{mspkcs7}
{\em PKCS | Microsoft Docs}.

  \verb|https://docs.microsoft.com/en-us/windows/desktop/seccertenroll/pkcs--7-attributes|
  \newblock Accessed: 2018-12-6.

\bibitem{pkcs12_struct}
{\em RFC 7292 - Page 10}.

  \verb|https://tools.ietf.org/html/rfc7292#page-10|
  \newblock Accessed: 2018-12-6.

\end{thebibliography}

\end{document}
